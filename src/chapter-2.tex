%! suppress = TooLargeSection


The execution plan deliberately refrains from specifying exact implementation details, instead it tries to explore
possibilities rather than fixed pathways.\ This flexibility is essential because the implementation details will vary
depending on the allocated budget.
\\
The computer science team does not require much funding, if any.\ All the funding is for the systems administration team
which will acquire various hardware required to set up the operations detailed in the preceding chapter.


%----------------------------------------------------------------------------------------------------------------------%
%----------------------------------------------------------------------------------------------------------------------%


\section{Considerations \& Assumptions}\label{sec:considerations-&-assumptions}
This section describes a few critical considerations that the university should carefully contemplate to avoid
developing unrealistic expectations from the club, the proposal also bases some assumptions upon these considerations
to reduce initial costs.\ These are as follows:

\begin{itemize}
    \item CSSAC is not a one-time investment, it has running costs and so an annual budget is to be expected.
    \item Servers consume considerable amounts of electrical power, the proposed servers assume that university receives
    subsidized electricity at cheap rates and also produces a portion of its electrical power through solar energy.\\
    Another assumption has been made about the power reliability on the campus, a runtime of not more than 2--4 minutes
    is to be expected from the proposed UPS solutions.\ If the university is equipped with generators, they should start
    delivering power within this time-frame.
\end{itemize}


%----------------------------------------------------------------------------------------------------------------------%
%----------------------------------------------------------------------------------------------------------------------%


