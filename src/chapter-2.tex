%! suppress = TooLargeSection


The execution plan deliberately refrains from specifying exact implementation details, instead it tries to explore
possibilities rather than fixed pathways.\ This flexibility is essential because the implementation details will vary
depending on the allocated budget.
\\
The computer science team does not require much funding, if any.\ All the funding is for the systems administration team
which will acquire various hardware required to set up the operations detailed in the preceding chapter.


%----------------------------------------------------------------------------------------------------------------------%
%----------------------------------------------------------------------------------------------------------------------%


\section{Considerations \& Assumptions}\label{sec:considerations-&-assumptions}
This section describes a few critical considerations that the university should carefully contemplate to avoid
developing unrealistic expectations from the club, the proposal also bases some assumptions upon these considerations
to reduce initial costs.\ These are as follows:

\begin{itemize}
    \item CSSAC is not a one-time investment, it has running costs and so an annual budget is to be expected.
    \item Servers consume considerable amounts of electrical power, the proposed servers assume that university receives
    subsidized electricity at cheap rates and also produces a portion of its electrical power through solar energy.\\
    Another assumption has been made about the power reliability on the campus, a runtime of not more than 2--4 minutes
    is to be expected from the proposed UPS solutions.\ If the university is equipped with generators, they should start
    delivering power within this time-frame.
\end{itemize}


%----------------------------------------------------------------------------------------------------------------------%
%----------------------------------------------------------------------------------------------------------------------%


\section{Requirements}\label{sec:requirements}
If the proposal is accepted, the club has the following requirements from the university:

%----------------------------------------------------------------------------------------------------------------------%

\subsection{Room Allotment}\label{subsec:room-allotment}

\subsubsection{Server Room}
When selecting a server room, several critical factors must be accounted for.\ The chosen room shall have proper
equipment to control temperature and humidity, too much humidity will cause condensation and too little will promote
electrostatic discharge.\ Ideally, the room shall not have any windows (or the windows should be covered) or be located
nearby any potential sources of water which might leak.\ Considering the weight of the servers and racks will increase
over time as more equipment is added over the upcoming years, it will be essential to assess the load-bearing capacity
of the floor to ensure it can safely accommodate the anticipated growth.

\subsubsection{Operations Room}
This room will serve as the home to the entire club.\ It will be the place to keep inventory of random parts such as
raid cards, hard drives, warranty documents, hardware projects by the CS team, etc.\ As suggested by the name of the
room, it will also be the place where the sysadmins will administer the servers and thus this room will require around
2--4 computers.\ The reason to have two separate rooms instead of sharing a common room is due to the fact that servers
are loud and long term exposure to such noise can cause hearing loss.

%----------------------------------------------------------------------------------------------------------------------%

\begin{table*}[t]
    \caption{Price List}
    \label{table:price-list}
    \begin{tblr}{colspec={|X[2]|X|}, hlines}
        \textbf{Component}                            & \textbf{Price (INR)} \\
        Dell Poweredge R820 (32GB DDR3, 4xE5-4620 v2) & 88,499               \\
        Dell Poweredge R730 (64GB DDR4, 2xE5-2680 v3) & 100,499              \\
        Dell Poweredge R710 (32GB DDR4, 2xX5650)      & 39,435               \\
        Intel Xeon E5-4650 V2 Processor               & 4,999                \\
        Intel Xeon E5-2696 V4 Processor               & 25,960               \\
        Any 24--48 port managed network switch        & 20,000               \\
        Any 42u Server Rack                           & 50,000               \\
        Any >3000VA UPS                               & 60,000               \\
    \end{tblr}
\end{table*}

%----------------------------------------------------------------------------------------------------------------------%

\subsection{Budget}\label{subsec:budget}
To estimate a budget, table~\ref{table:price-list} lists shortlists some hardware and an estimate as to what they might
cost, prices are sourced from the website \url{serverbasket.com}
\\
I would strongly recommend at least 2 physical servers for reasons such as redundancy and isolation but mostly
convenience.\ The Dell R730 is a newer generation server compared to the R820 and supports DDR4 memory and newer
generation processors.\ The R820 only supports DDR3 memory and relatively older processors, but it offers 4 CPU sockets.
Either one of these servers is suitable to act as our primary server but the R730 would be my choice to save
electricity.\ A cheaper R710 could be used to virtualize the firewall, running the firewall on physically separate
hardware offers convenience while experimenting with the primary server.\ A dedicated firewall device could also be
considered.
\\\\
All that said, a budget of about \textbf{\large3--4 Lakh INR} is to be expected for the following items:
\begin{itemize}
    \item Dell Poweredge R730 (128GB DDR4, 2xE5-2696 v4)
    \item Dell Poweredge R710 (32GB DDR4, 2xX5650)
    \item 2x8TB SATA HDD budget upto 36,000 INR
    \item 24/48 port network switch budget upto 20,000 INR
    \item 42u server rack budget upto 50,000 INR
    \item UPS budget upto 60,000 INR
\end{itemize}
\\
This is by no means the final budget, instead it offers a ballpark estimation to allow the reader to decide whether to
accept this proposal or not.\ If the proposal is accepted, I can start requesting quotes from various vendors on
behalf of the university both online and offline (Nehru Place market) and prepare a proper budget.