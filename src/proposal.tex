%! Author = Sujal Singh
%! Date = 9/25/23

% Preamble
\documentclass[11pt,twocolumn,oneside]{book}

\title{\textbf{Proposal for the Establishment of CSSAC: \\ Computer Science \& Systems Administration Club}}
\author{Sujal Singh \\ University School of Automation \& Robotics}
\date{Dated: September 25, 2023}

% Packages
\usepackage{amsmath}
\usepackage{lipsum}
\usepackage[T1]{fontenc}
\usepackage[utf8]{inputenc}
\usepackage[a4paper,left=2cm,right=2cm,top=3cm,bottom=3cm]{geometry}
\usepackage[
    pdftitle={CSSAC Proposal},
    pdfsubject={CSSAC Proposal},
    pdfauthor={Sujal Singh},
    pdfdisplaydoctitle,
    hidelinks,
]{hyperref}
\usepackage{color}

% Document
\begin{document}
    \maketitle
    \tableofcontents


    \chapter{Introduction: The What and Why}\label{ch:introduction}


    \section{Abstract}\label{sec:abstract}
    This proposal requests the establishment of a Computer Science \& Systems Administration Club (CSSAC), an
    exceptionally unique and innovative endeavor, as the existence of such clubs in Indian universities is rare, if not
    non-existent.
    Such a club offers unparalleled learning experiences to members, has practical usefulness to virtually all other
    clubs and has the potential to offer facilities typically only found in the Indian Institutes of Technology and
    various leading universities abroad such as MIT, Stanford, University of California and The University of Waterloo.
    \\
    All details necessary for the creation of CSSAC are outlined in this proposal, including requirements from the
    university, and it also demonstrates that the author is capable of leading the technical aspects of this initiative.


    \section{Vision}\label{sec:vision}
    The primary goal of CSSAC will be to cultivate a community of technically competent members, who possess a strong
    theoretical foundation in computer science but also have the ability to practically apply that knowledge to
    real-world problems.\ After all, we're soon to be engineers.
    CSSAC places no emphasis on tasks aimed solely at securing better job prospects, but only on the goal of
    utilizing our time at the university to become better engineers, and yet it'll allow CSSAC members to distinguish
    themselves from others in a pool of equally talented candidates.\ CSSAC also strives to promote
    \href{https://wikipedia.org/wiki/Hacker_culture}{\color{blue}\underline{hacker culture}} (not to be confused with
    cracker) throughout the university.
    In practical terms, this translates into having two subteams (not necessarily mutually exclusive), as implied by the
    name of the club.\ These subteams will undertake two different sets of problems described in their respective
    sections, one dealing with mostly abstract problems and the other applying the learnings from the aforementioned
    team to practical purposes with fruitful outcomes.

    \section{\large Systems Administration Team}\label{sec:the-systems-administration-team}
    This systems administration team is the one dealing with practical problems.\ This team's goal will be to create
    and maintain an on-premise cloud service that provides various services to other clubs, faculty, members and
    individual students.
    CSSAC will also reserve a portion of the resources to deploy services for
    everyone\footnote{Limited to those who register for an account and available capacity.} in the university.
    Initially, CSSAC might reserve all resources to deploy services that are available to everyone in the university.
\end{document}