\section{Abstract}\label{sec:abstract}
This proposal requests the establishment of a Computer Science \& Systems Administration Club (CSSAC), an
exceptionally unique and innovative endeavor, as the existence of such clubs in Indian universities is rare, if not
non-existent.
Such a club offers unparalleled learning experiences to members, has practical usefulness to virtually all other
clubs and has the potential to offer facilities typically only found in the Indian Institutes of Technology and
various leading universities abroad such as MIT, Stanford, University of California and The University of Waterloo.
\\
All details necessary for the creation of CSSAC are outlined in this proposal, including requirements from the
university, and it also demonstrates that the author is capable of leading the technical aspects of this initiative.


%------------------------------------------------------------------------------------------------------------------%
%------------------------------------------------------------------------------------------------------------------%


\section{Vision}\label{sec:vision}
\emph{``What I cannot create, I do not understand.''}{\par\hfill\small{---Richard Feynman}}
\\\\
At the very heart of this club is the goal to learn by doing, to obtain in-depth knowledge, to cultivate a community
of technically competent members who possess a strong theoretical foundation in computer science but also have the
ability to practically apply that knowledge to real-world problems.\ After all, we're soon to be engineers.
CSSAC places no emphasis on tasks aimed solely at securing better job prospects, but only on the goal of
utilizing our time at the university to become better engineers, and yet it'll allow members to distinguish
themselves from others in a pool of equally talented candidates.\ CSSAC also strives to promote
\href{https://wikipedia.org/wiki/Hacker_culture}{\color{blue}hacker culture} (not to be confused with
cracker) throughout the university.
In practical terms, this translates into having two subteams (not necessarily mutually exclusive), as implied by the
name of the club.\ These subteams will undertake two different sets of problems described in their respective
sections, a computer science team dealing with mostly abstract problems and the systems administration team applying
the learnings from the aforementioned team to practical purposes with fruitful outcomes.


%------------------------------------------------------------------------------------------------------------------%
%------------------------------------------------------------------------------------------------------------------%


%! suppress = TooLargeSection


\section{\large Systems Administration Team}\label{sec:systems-administration-team}
The systems administration team focuses on addressing practical problems.\ It's primary objectives are as follows:
\begin{enumerate}
    \item \textbf{On-premise Cloud:} The systems administration team will design, establish and maintain an
    on-premise private cloud offering available as a service to members, other clubs, faculty and individual
    students.\ This also lays a foundation for significantly more ambitious projects in the future such as providing
    high-performance computing (HPC) to research projects in the university.
    \item \textbf{Services:} CSSAC will reserve a portion of the resources to deploy services for the
    entire\footnote{Limited to those who register for an account and available capacity.} university.\ This will
    include services like providing cloud storage, source code hosting, mail services and mailing lists, static web
    hosting, etc.\ Initially, we might reserve all resources to deploy services that are available to everyone in
    the university.
\end{enumerate}
The exact implementation details are not discussed in this proposal since this will mostly be dictated by the
hardware provided to the club.

%------------------------------------------------------------------------------------------------------------------%

\subsection[Symbiotic Relationship with Other Clubs]{Symbiotic Relationship \\ with Other Clubs}
\label{subsec:symbiotic-relationship-with-other-clubs}
This club has the potential to offer significant utility to other clubs and offer them the opportunity to develop
various innovative projects in collaboration such that these projects are mutually beneficial to all parties
involved.\ Several such scenarios are described in the following sections:

\subsubsection{IoSC, GDSC, SDC, ACM and IEEE}
Many clubs in our university have various web development and AI teams, all of which stand to gain immense benefits
from the compute services offered by CSSAC.\ We could also offer GPU compute services in the future to be able to
train machine learning models and other applications which heavily benefit from GPU compute.
\\
CSSAC also stands to benefit from these clubs; for example, instead of creating our own web development team we
could offload such tasks to the web development teams of these clubs\@.

\subsubsection{Karuyantra Club}
The Karuyantra club could offer us 3D printing services to be able to produce various makeshift solutions in an
example scenario such as when a network card's bracket will not fit the chassis of the server.\ They can also
work on a project to develop a robot which can remotely perform tasks in a data center that would normally require a
human's physical presence such as replacing faulty hard drives or periodic reapplication of thermal paste on CPUs.

%------------------------------------------------------------------------------------------------------------------%

\subsection{Proof of Concept}\label{subsec:proof-of-concept}

While the idea for the club appears novel, it heavily draws inspiration from entities that have existed at various
institutes worldwide since the early days of the Internet.\ This is not a drawback; rather, it proves the need for
the existence of this club exists, and it's implementation is achievable in an academic setting.
\\
Here follow some examples of existing entities which perform operations akin to those of the systems administration
team of CSSAC:

\begin{enumerate}
    \item \textbf{Computer Science Club, University of Waterloo}\\
    This is an entirely student-run operation that has been able to accumulate 4 racks full of servers over the
    years and has a cloud offering for members at just \$2 per term.\ Moreover, they've also deployed many services
    both for their members and some for the public good such as a Linux mirror.\ HPC is handled by other bodies at
    this institute.\ Read more at:\\
    \href{https://wiki.csclub.uwaterloo.ca/}{\color{blue}\url{https://wiki.csclub.uwaterloo.ca/}}

    \item \textbf{Computer Services Centre, IIT Delhi}\\
    IIT Delhi has developed their very own cloud offering named ``Baadal'', which allows users to spawn resources
    via a web interface also developed at IITD.\ They plan to put to use the underutilized computers in most
    computer labs and still maintain reliability by running redundant copies of virtual machines on different
    physical nodes, which is something we should also experiment with.\ They also have HPC facilities available,
    read more at:\\
    \href{https://csc.iitd.ac.in/}{\color{blue}\url{https://csc.iitd.ac.in/}}
\end{enumerate}

Additionally, I am also in the process of deploying several servers at home, which has provided me with innumerable
learning opportunities.

%------------------------------------------------------------------------------------------------------------------%

\subsection{Outcomes}\label{subsec:outcomes}
This section makes an attempt at answering the question \emph{``What does this accomplish?''} to fairly
non-technical readers, because technical readers might already be familiar with the potential outcomes of such an
undertaking.

\subsubsection{Examples}
The easiest explanation here would be to just enumerate some examples, and so, here follow some tasks that the
systems administration team will be concerned with:

\begin{itemize}
    \item Faculty XYZ needs to deploy a static homepage.\ For example, IIT Delhi allows faculty and students to
    deploy a homepage under the following URL pattern:\\
    \href{https://web.iitd.ac.in/~name}{\color{blue}\url{https://web.iitd.ac.in/~name}}.
    \item Imagine as a student the university grants you certain services, for example, a certain amount of storage
    accessible from all university computers upon login with your credentials (this also hints towards a central
    authentication server) or even accessible through the Internet, such services would be operated by the systems
    administration team.
    \item In cooperation with the team currently administering the university network, we could implement various
    methods to isolate parts of the network from one another, deploy VPN services, implement security best
    practices, etc.\ I've also noticed an issue with the current network wherein I speculate that the server
    responsible for issuing addresses to devices runs out of said addresses in the allowed range, a student
    bypassing this server by manually filling in their address is granted access to the network while other students
    are not, we might be able to fix such problems.
    \item Many premier institutes often host a mirror for various Linux distributions, such a service often causes
    global appreciation for the university and also benefits students by offering high speed downloads.\ Such a
    service can be maintained by this club given additional storage is provided.
\end{itemize}

The list could go on, but the fundamental point of all of these outcomes is that having such facilities on-campus,
removes the \emph{``magic''} from data centers and allows members of the club to have hands-on experience with the
processes that happen behind-the-scenes to the run the modern world.\\


%------------------------------------------------------------------------------------------------------------------%
%------------------------------------------------------------------------------------------------------------------%


